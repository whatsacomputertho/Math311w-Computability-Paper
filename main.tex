\documentclass{article}
\usepackage[utf8]{inputenc}
\usepackage{biblatex}

\title{Computability: Proof by Construction and Disproof by Contradiction}
\author{Oluwafunke Alliyu, Ethan Balcik, Paul John Balderston, Sen Zhu}
\date{April 2021}

\begin{document}

\maketitle

\section{Introduction}
In the age of digital communications, it is difficult to imagine how, in a world without digital computers, one might work toward the development of one.  As a result, often overlooked are the initial developments which led to today's digital age.  Nevertheless, the emergence of digital computers would not be possible without initial, groundbreaking developments in mathematical logic and information theory from significant engineers and mathematicians like Alan Turing and Claude Shannon.  Throughout this paper, we discuss computability in a rigorous, self-contained manner - both its proof by construction and its disproof by contradiction, with examples of each.  Furthermore, we aim to provide our readers with some historical insight into the development of these methods in order to build both an appreciation of their significance, and of the current challenges researchers face as attempts are made to further progress in theoretical computer science \cite{1}.

\section{Background}
Next step is the background!

\section{Turing Machines}
\section{Proof of Computability by Construction}
\section{Disproof of Computability by Contradiction}
\section{Conclusions}
\section{References}
\begin{thebibliography}{100}
\bibitem{1} National Research Council. 1999. Funding a Revolution: Government Support for Computing Research. Washington, DC: The National Academies Press. https://doi.org/10.17226/6323.
\end{thebibliography}

\end{document}
